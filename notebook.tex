
% Default to the notebook output style

    


% Inherit from the specified cell style.




    
\documentclass[11pt]{article}

    
    
    \usepackage[T1]{fontenc}
    % Nicer default font (+ math font) than Computer Modern for most use cases
    \usepackage{mathpazo}

    % Basic figure setup, for now with no caption control since it's done
    % automatically by Pandoc (which extracts ![](path) syntax from Markdown).
    \usepackage{graphicx}
    % We will generate all images so they have a width \maxwidth. This means
    % that they will get their normal width if they fit onto the page, but
    % are scaled down if they would overflow the margins.
    \makeatletter
    \def\maxwidth{\ifdim\Gin@nat@width>\linewidth\linewidth
    \else\Gin@nat@width\fi}
    \makeatother
    \let\Oldincludegraphics\includegraphics
    % Set max figure width to be 80% of text width, for now hardcoded.
    \renewcommand{\includegraphics}[1]{\Oldincludegraphics[width=.8\maxwidth]{#1}}
    % Ensure that by default, figures have no caption (until we provide a
    % proper Figure object with a Caption API and a way to capture that
    % in the conversion process - todo).
    \usepackage{caption}
    \DeclareCaptionLabelFormat{nolabel}{}
    \captionsetup{labelformat=nolabel}

    \usepackage{adjustbox} % Used to constrain images to a maximum size 
    \usepackage{xcolor} % Allow colors to be defined
    \usepackage{enumerate} % Needed for markdown enumerations to work
    \usepackage{geometry} % Used to adjust the document margins
    \usepackage{amsmath} % Equations
    \usepackage{amssymb} % Equations
    \usepackage{textcomp} % defines textquotesingle
    % Hack from http://tex.stackexchange.com/a/47451/13684:
    \AtBeginDocument{%
        \def\PYZsq{\textquotesingle}% Upright quotes in Pygmentized code
    }
    \usepackage{upquote} % Upright quotes for verbatim code
    \usepackage{eurosym} % defines \euro
    \usepackage[mathletters]{ucs} % Extended unicode (utf-8) support
    \usepackage[utf8x]{inputenc} % Allow utf-8 characters in the tex document
    \usepackage{fancyvrb} % verbatim replacement that allows latex
    \usepackage{grffile} % extends the file name processing of package graphics 
                         % to support a larger range 
    % The hyperref package gives us a pdf with properly built
    % internal navigation ('pdf bookmarks' for the table of contents,
    % internal cross-reference links, web links for URLs, etc.)
    \usepackage{hyperref}
    \usepackage{longtable} % longtable support required by pandoc >1.10
    \usepackage{booktabs}  % table support for pandoc > 1.12.2
    \usepackage[inline]{enumitem} % IRkernel/repr support (it uses the enumerate* environment)
    \usepackage[normalem]{ulem} % ulem is needed to support strikethroughs (\sout)
                                % normalem makes italics be italics, not underlines
    

    
    
    % Colors for the hyperref package
    \definecolor{urlcolor}{rgb}{0,.145,.698}
    \definecolor{linkcolor}{rgb}{.71,0.21,0.01}
    \definecolor{citecolor}{rgb}{.12,.54,.11}

    % ANSI colors
    \definecolor{ansi-black}{HTML}{3E424D}
    \definecolor{ansi-black-intense}{HTML}{282C36}
    \definecolor{ansi-red}{HTML}{E75C58}
    \definecolor{ansi-red-intense}{HTML}{B22B31}
    \definecolor{ansi-green}{HTML}{00A250}
    \definecolor{ansi-green-intense}{HTML}{007427}
    \definecolor{ansi-yellow}{HTML}{DDB62B}
    \definecolor{ansi-yellow-intense}{HTML}{B27D12}
    \definecolor{ansi-blue}{HTML}{208FFB}
    \definecolor{ansi-blue-intense}{HTML}{0065CA}
    \definecolor{ansi-magenta}{HTML}{D160C4}
    \definecolor{ansi-magenta-intense}{HTML}{A03196}
    \definecolor{ansi-cyan}{HTML}{60C6C8}
    \definecolor{ansi-cyan-intense}{HTML}{258F8F}
    \definecolor{ansi-white}{HTML}{C5C1B4}
    \definecolor{ansi-white-intense}{HTML}{A1A6B2}

    % commands and environments needed by pandoc snippets
    % extracted from the output of `pandoc -s`
    \providecommand{\tightlist}{%
      \setlength{\itemsep}{0pt}\setlength{\parskip}{0pt}}
    \DefineVerbatimEnvironment{Highlighting}{Verbatim}{commandchars=\\\{\}}
    % Add ',fontsize=\small' for more characters per line
    \newenvironment{Shaded}{}{}
    \newcommand{\KeywordTok}[1]{\textcolor[rgb]{0.00,0.44,0.13}{\textbf{{#1}}}}
    \newcommand{\DataTypeTok}[1]{\textcolor[rgb]{0.56,0.13,0.00}{{#1}}}
    \newcommand{\DecValTok}[1]{\textcolor[rgb]{0.25,0.63,0.44}{{#1}}}
    \newcommand{\BaseNTok}[1]{\textcolor[rgb]{0.25,0.63,0.44}{{#1}}}
    \newcommand{\FloatTok}[1]{\textcolor[rgb]{0.25,0.63,0.44}{{#1}}}
    \newcommand{\CharTok}[1]{\textcolor[rgb]{0.25,0.44,0.63}{{#1}}}
    \newcommand{\StringTok}[1]{\textcolor[rgb]{0.25,0.44,0.63}{{#1}}}
    \newcommand{\CommentTok}[1]{\textcolor[rgb]{0.38,0.63,0.69}{\textit{{#1}}}}
    \newcommand{\OtherTok}[1]{\textcolor[rgb]{0.00,0.44,0.13}{{#1}}}
    \newcommand{\AlertTok}[1]{\textcolor[rgb]{1.00,0.00,0.00}{\textbf{{#1}}}}
    \newcommand{\FunctionTok}[1]{\textcolor[rgb]{0.02,0.16,0.49}{{#1}}}
    \newcommand{\RegionMarkerTok}[1]{{#1}}
    \newcommand{\ErrorTok}[1]{\textcolor[rgb]{1.00,0.00,0.00}{\textbf{{#1}}}}
    \newcommand{\NormalTok}[1]{{#1}}
    
    % Additional commands for more recent versions of Pandoc
    \newcommand{\ConstantTok}[1]{\textcolor[rgb]{0.53,0.00,0.00}{{#1}}}
    \newcommand{\SpecialCharTok}[1]{\textcolor[rgb]{0.25,0.44,0.63}{{#1}}}
    \newcommand{\VerbatimStringTok}[1]{\textcolor[rgb]{0.25,0.44,0.63}{{#1}}}
    \newcommand{\SpecialStringTok}[1]{\textcolor[rgb]{0.73,0.40,0.53}{{#1}}}
    \newcommand{\ImportTok}[1]{{#1}}
    \newcommand{\DocumentationTok}[1]{\textcolor[rgb]{0.73,0.13,0.13}{\textit{{#1}}}}
    \newcommand{\AnnotationTok}[1]{\textcolor[rgb]{0.38,0.63,0.69}{\textbf{\textit{{#1}}}}}
    \newcommand{\CommentVarTok}[1]{\textcolor[rgb]{0.38,0.63,0.69}{\textbf{\textit{{#1}}}}}
    \newcommand{\VariableTok}[1]{\textcolor[rgb]{0.10,0.09,0.49}{{#1}}}
    \newcommand{\ControlFlowTok}[1]{\textcolor[rgb]{0.00,0.44,0.13}{\textbf{{#1}}}}
    \newcommand{\OperatorTok}[1]{\textcolor[rgb]{0.40,0.40,0.40}{{#1}}}
    \newcommand{\BuiltInTok}[1]{{#1}}
    \newcommand{\ExtensionTok}[1]{{#1}}
    \newcommand{\PreprocessorTok}[1]{\textcolor[rgb]{0.74,0.48,0.00}{{#1}}}
    \newcommand{\AttributeTok}[1]{\textcolor[rgb]{0.49,0.56,0.16}{{#1}}}
    \newcommand{\InformationTok}[1]{\textcolor[rgb]{0.38,0.63,0.69}{\textbf{\textit{{#1}}}}}
    \newcommand{\WarningTok}[1]{\textcolor[rgb]{0.38,0.63,0.69}{\textbf{\textit{{#1}}}}}
    
    
    % Define a nice break command that doesn't care if a line doesn't already
    % exist.
    \def\br{\hspace*{\fill} \\* }
    % Math Jax compatability definitions
    \def\gt{>}
    \def\lt{<}
    % Document parameters
    \title{0. Tutorial\_inicio}
    
    
    

    % Pygments definitions
    
\makeatletter
\def\PY@reset{\let\PY@it=\relax \let\PY@bf=\relax%
    \let\PY@ul=\relax \let\PY@tc=\relax%
    \let\PY@bc=\relax \let\PY@ff=\relax}
\def\PY@tok#1{\csname PY@tok@#1\endcsname}
\def\PY@toks#1+{\ifx\relax#1\empty\else%
    \PY@tok{#1}\expandafter\PY@toks\fi}
\def\PY@do#1{\PY@bc{\PY@tc{\PY@ul{%
    \PY@it{\PY@bf{\PY@ff{#1}}}}}}}
\def\PY#1#2{\PY@reset\PY@toks#1+\relax+\PY@do{#2}}

\expandafter\def\csname PY@tok@w\endcsname{\def\PY@tc##1{\textcolor[rgb]{0.73,0.73,0.73}{##1}}}
\expandafter\def\csname PY@tok@c\endcsname{\let\PY@it=\textit\def\PY@tc##1{\textcolor[rgb]{0.25,0.50,0.50}{##1}}}
\expandafter\def\csname PY@tok@cp\endcsname{\def\PY@tc##1{\textcolor[rgb]{0.74,0.48,0.00}{##1}}}
\expandafter\def\csname PY@tok@k\endcsname{\let\PY@bf=\textbf\def\PY@tc##1{\textcolor[rgb]{0.00,0.50,0.00}{##1}}}
\expandafter\def\csname PY@tok@kp\endcsname{\def\PY@tc##1{\textcolor[rgb]{0.00,0.50,0.00}{##1}}}
\expandafter\def\csname PY@tok@kt\endcsname{\def\PY@tc##1{\textcolor[rgb]{0.69,0.00,0.25}{##1}}}
\expandafter\def\csname PY@tok@o\endcsname{\def\PY@tc##1{\textcolor[rgb]{0.40,0.40,0.40}{##1}}}
\expandafter\def\csname PY@tok@ow\endcsname{\let\PY@bf=\textbf\def\PY@tc##1{\textcolor[rgb]{0.67,0.13,1.00}{##1}}}
\expandafter\def\csname PY@tok@nb\endcsname{\def\PY@tc##1{\textcolor[rgb]{0.00,0.50,0.00}{##1}}}
\expandafter\def\csname PY@tok@nf\endcsname{\def\PY@tc##1{\textcolor[rgb]{0.00,0.00,1.00}{##1}}}
\expandafter\def\csname PY@tok@nc\endcsname{\let\PY@bf=\textbf\def\PY@tc##1{\textcolor[rgb]{0.00,0.00,1.00}{##1}}}
\expandafter\def\csname PY@tok@nn\endcsname{\let\PY@bf=\textbf\def\PY@tc##1{\textcolor[rgb]{0.00,0.00,1.00}{##1}}}
\expandafter\def\csname PY@tok@ne\endcsname{\let\PY@bf=\textbf\def\PY@tc##1{\textcolor[rgb]{0.82,0.25,0.23}{##1}}}
\expandafter\def\csname PY@tok@nv\endcsname{\def\PY@tc##1{\textcolor[rgb]{0.10,0.09,0.49}{##1}}}
\expandafter\def\csname PY@tok@no\endcsname{\def\PY@tc##1{\textcolor[rgb]{0.53,0.00,0.00}{##1}}}
\expandafter\def\csname PY@tok@nl\endcsname{\def\PY@tc##1{\textcolor[rgb]{0.63,0.63,0.00}{##1}}}
\expandafter\def\csname PY@tok@ni\endcsname{\let\PY@bf=\textbf\def\PY@tc##1{\textcolor[rgb]{0.60,0.60,0.60}{##1}}}
\expandafter\def\csname PY@tok@na\endcsname{\def\PY@tc##1{\textcolor[rgb]{0.49,0.56,0.16}{##1}}}
\expandafter\def\csname PY@tok@nt\endcsname{\let\PY@bf=\textbf\def\PY@tc##1{\textcolor[rgb]{0.00,0.50,0.00}{##1}}}
\expandafter\def\csname PY@tok@nd\endcsname{\def\PY@tc##1{\textcolor[rgb]{0.67,0.13,1.00}{##1}}}
\expandafter\def\csname PY@tok@s\endcsname{\def\PY@tc##1{\textcolor[rgb]{0.73,0.13,0.13}{##1}}}
\expandafter\def\csname PY@tok@sd\endcsname{\let\PY@it=\textit\def\PY@tc##1{\textcolor[rgb]{0.73,0.13,0.13}{##1}}}
\expandafter\def\csname PY@tok@si\endcsname{\let\PY@bf=\textbf\def\PY@tc##1{\textcolor[rgb]{0.73,0.40,0.53}{##1}}}
\expandafter\def\csname PY@tok@se\endcsname{\let\PY@bf=\textbf\def\PY@tc##1{\textcolor[rgb]{0.73,0.40,0.13}{##1}}}
\expandafter\def\csname PY@tok@sr\endcsname{\def\PY@tc##1{\textcolor[rgb]{0.73,0.40,0.53}{##1}}}
\expandafter\def\csname PY@tok@ss\endcsname{\def\PY@tc##1{\textcolor[rgb]{0.10,0.09,0.49}{##1}}}
\expandafter\def\csname PY@tok@sx\endcsname{\def\PY@tc##1{\textcolor[rgb]{0.00,0.50,0.00}{##1}}}
\expandafter\def\csname PY@tok@m\endcsname{\def\PY@tc##1{\textcolor[rgb]{0.40,0.40,0.40}{##1}}}
\expandafter\def\csname PY@tok@gh\endcsname{\let\PY@bf=\textbf\def\PY@tc##1{\textcolor[rgb]{0.00,0.00,0.50}{##1}}}
\expandafter\def\csname PY@tok@gu\endcsname{\let\PY@bf=\textbf\def\PY@tc##1{\textcolor[rgb]{0.50,0.00,0.50}{##1}}}
\expandafter\def\csname PY@tok@gd\endcsname{\def\PY@tc##1{\textcolor[rgb]{0.63,0.00,0.00}{##1}}}
\expandafter\def\csname PY@tok@gi\endcsname{\def\PY@tc##1{\textcolor[rgb]{0.00,0.63,0.00}{##1}}}
\expandafter\def\csname PY@tok@gr\endcsname{\def\PY@tc##1{\textcolor[rgb]{1.00,0.00,0.00}{##1}}}
\expandafter\def\csname PY@tok@ge\endcsname{\let\PY@it=\textit}
\expandafter\def\csname PY@tok@gs\endcsname{\let\PY@bf=\textbf}
\expandafter\def\csname PY@tok@gp\endcsname{\let\PY@bf=\textbf\def\PY@tc##1{\textcolor[rgb]{0.00,0.00,0.50}{##1}}}
\expandafter\def\csname PY@tok@go\endcsname{\def\PY@tc##1{\textcolor[rgb]{0.53,0.53,0.53}{##1}}}
\expandafter\def\csname PY@tok@gt\endcsname{\def\PY@tc##1{\textcolor[rgb]{0.00,0.27,0.87}{##1}}}
\expandafter\def\csname PY@tok@err\endcsname{\def\PY@bc##1{\setlength{\fboxsep}{0pt}\fcolorbox[rgb]{1.00,0.00,0.00}{1,1,1}{\strut ##1}}}
\expandafter\def\csname PY@tok@kc\endcsname{\let\PY@bf=\textbf\def\PY@tc##1{\textcolor[rgb]{0.00,0.50,0.00}{##1}}}
\expandafter\def\csname PY@tok@kd\endcsname{\let\PY@bf=\textbf\def\PY@tc##1{\textcolor[rgb]{0.00,0.50,0.00}{##1}}}
\expandafter\def\csname PY@tok@kn\endcsname{\let\PY@bf=\textbf\def\PY@tc##1{\textcolor[rgb]{0.00,0.50,0.00}{##1}}}
\expandafter\def\csname PY@tok@kr\endcsname{\let\PY@bf=\textbf\def\PY@tc##1{\textcolor[rgb]{0.00,0.50,0.00}{##1}}}
\expandafter\def\csname PY@tok@bp\endcsname{\def\PY@tc##1{\textcolor[rgb]{0.00,0.50,0.00}{##1}}}
\expandafter\def\csname PY@tok@fm\endcsname{\def\PY@tc##1{\textcolor[rgb]{0.00,0.00,1.00}{##1}}}
\expandafter\def\csname PY@tok@vc\endcsname{\def\PY@tc##1{\textcolor[rgb]{0.10,0.09,0.49}{##1}}}
\expandafter\def\csname PY@tok@vg\endcsname{\def\PY@tc##1{\textcolor[rgb]{0.10,0.09,0.49}{##1}}}
\expandafter\def\csname PY@tok@vi\endcsname{\def\PY@tc##1{\textcolor[rgb]{0.10,0.09,0.49}{##1}}}
\expandafter\def\csname PY@tok@vm\endcsname{\def\PY@tc##1{\textcolor[rgb]{0.10,0.09,0.49}{##1}}}
\expandafter\def\csname PY@tok@sa\endcsname{\def\PY@tc##1{\textcolor[rgb]{0.73,0.13,0.13}{##1}}}
\expandafter\def\csname PY@tok@sb\endcsname{\def\PY@tc##1{\textcolor[rgb]{0.73,0.13,0.13}{##1}}}
\expandafter\def\csname PY@tok@sc\endcsname{\def\PY@tc##1{\textcolor[rgb]{0.73,0.13,0.13}{##1}}}
\expandafter\def\csname PY@tok@dl\endcsname{\def\PY@tc##1{\textcolor[rgb]{0.73,0.13,0.13}{##1}}}
\expandafter\def\csname PY@tok@s2\endcsname{\def\PY@tc##1{\textcolor[rgb]{0.73,0.13,0.13}{##1}}}
\expandafter\def\csname PY@tok@sh\endcsname{\def\PY@tc##1{\textcolor[rgb]{0.73,0.13,0.13}{##1}}}
\expandafter\def\csname PY@tok@s1\endcsname{\def\PY@tc##1{\textcolor[rgb]{0.73,0.13,0.13}{##1}}}
\expandafter\def\csname PY@tok@mb\endcsname{\def\PY@tc##1{\textcolor[rgb]{0.40,0.40,0.40}{##1}}}
\expandafter\def\csname PY@tok@mf\endcsname{\def\PY@tc##1{\textcolor[rgb]{0.40,0.40,0.40}{##1}}}
\expandafter\def\csname PY@tok@mh\endcsname{\def\PY@tc##1{\textcolor[rgb]{0.40,0.40,0.40}{##1}}}
\expandafter\def\csname PY@tok@mi\endcsname{\def\PY@tc##1{\textcolor[rgb]{0.40,0.40,0.40}{##1}}}
\expandafter\def\csname PY@tok@il\endcsname{\def\PY@tc##1{\textcolor[rgb]{0.40,0.40,0.40}{##1}}}
\expandafter\def\csname PY@tok@mo\endcsname{\def\PY@tc##1{\textcolor[rgb]{0.40,0.40,0.40}{##1}}}
\expandafter\def\csname PY@tok@ch\endcsname{\let\PY@it=\textit\def\PY@tc##1{\textcolor[rgb]{0.25,0.50,0.50}{##1}}}
\expandafter\def\csname PY@tok@cm\endcsname{\let\PY@it=\textit\def\PY@tc##1{\textcolor[rgb]{0.25,0.50,0.50}{##1}}}
\expandafter\def\csname PY@tok@cpf\endcsname{\let\PY@it=\textit\def\PY@tc##1{\textcolor[rgb]{0.25,0.50,0.50}{##1}}}
\expandafter\def\csname PY@tok@c1\endcsname{\let\PY@it=\textit\def\PY@tc##1{\textcolor[rgb]{0.25,0.50,0.50}{##1}}}
\expandafter\def\csname PY@tok@cs\endcsname{\let\PY@it=\textit\def\PY@tc##1{\textcolor[rgb]{0.25,0.50,0.50}{##1}}}

\def\PYZbs{\char`\\}
\def\PYZus{\char`\_}
\def\PYZob{\char`\{}
\def\PYZcb{\char`\}}
\def\PYZca{\char`\^}
\def\PYZam{\char`\&}
\def\PYZlt{\char`\<}
\def\PYZgt{\char`\>}
\def\PYZsh{\char`\#}
\def\PYZpc{\char`\%}
\def\PYZdl{\char`\$}
\def\PYZhy{\char`\-}
\def\PYZsq{\char`\'}
\def\PYZdq{\char`\"}
\def\PYZti{\char`\~}
% for compatibility with earlier versions
\def\PYZat{@}
\def\PYZlb{[}
\def\PYZrb{]}
\makeatother


    % Exact colors from NB
    \definecolor{incolor}{rgb}{0.0, 0.0, 0.5}
    \definecolor{outcolor}{rgb}{0.545, 0.0, 0.0}



    
    % Prevent overflowing lines due to hard-to-break entities
    \sloppy 
    % Setup hyperref package
    \hypersetup{
      breaklinks=true,  % so long urls are correctly broken across lines
      colorlinks=true,
      urlcolor=urlcolor,
      linkcolor=linkcolor,
      citecolor=citecolor,
      }
    % Slightly bigger margins than the latex defaults
    
    \geometry{verbose,tmargin=1in,bmargin=1in,lmargin=1in,rmargin=1in}
    
    

    \begin{document}
    
    
    \maketitle
    
    

    
    \section{¿Qué son los notebooks de Jupyter
?}\label{quuxe9-son-los-notebooks-de-jupyter}

Son una aplicación web open source que permite crear documentos con
código, texto enriquecido y visualizaciones. Ideal para crear y
compartir documentos que contienen ecuaciones, texto junto a código
(Python, R, Ruby, Sage, Haskell, Go, Spark ..) Permiten que el trabajo
sea reproducible y más interactivo.

Ventajas

\begin{itemize}
\tightlist
\item
  Elige el lenguaje.
\item
  Fácil de compartir.
\item
  Widgets interactivos.
\end{itemize}

Se pueden poner ecuaciones en una celda de tipo Markdown como esta

\[\int_0^\infty e^{-x^2} dx=\frac{\sqrt{\pi}}{2}\]

    En este notebook se abordarán las siguientes temáticas:

\begin{verbatim}
0. Documentación en python
1. Variables y tipos
2. Operadores
\end{verbatim}

    \subsection{0. Documentación en
python}\label{documentaciuxf3n-en-python}

    Para acceder a la documentación de python y obtener ayuda sobre los
diferentes módulos o funciones se hace uso del simbolo de interrogación
(?). De esta manera se accederá a una introducción y revisión de las
características de Ipython

    \begin{Verbatim}[commandchars=\\\{\}]
{\color{incolor}In [{\color{incolor}3}]:} \PY{o}{?}
\end{Verbatim}


    Con el comando help() se accede al sistema de ayuda propio de Python

    \begin{Verbatim}[commandchars=\\\{\}]
{\color{incolor}In [{\color{incolor}20}]:} \PY{n}{help}\PY{p}{(}\PY{p}{)}
\end{Verbatim}


    \begin{Verbatim}[commandchars=\\\{\}]

Welcome to Python 3.6's help utility!

If this is your first time using Python, you should definitely check out
the tutorial on the Internet at http://docs.python.org/3.6/tutorial/.

Enter the name of any module, keyword, or topic to get help on writing
Python programs and using Python modules.  To quit this help utility and
return to the interpreter, just type "quit".

To get a list of available modules, keywords, symbols, or topics, type
"modules", "keywords", "symbols", or "topics".  Each module also comes
with a one-line summary of what it does; to list the modules whose name
or summary contain a given string such as "spam", type "modules spam".

help> quit

You are now leaving help and returning to the Python interpreter.
If you want to ask for help on a particular object directly from the
interpreter, you can type "help(object)".  Executing "help('string')"
has the same effect as typing a particular string at the help> prompt.

    \end{Verbatim}

    En jupyter se cuenta con funciones "mágicas" que proporcionan una serie
de funciones que permiten tener control sobre el comportamiento de
Ipython, por ejemplo, el tiempo de ejecución de una línea o una celda.
Las funciones "mágicas" tienen como prefijo el simbolo \% o \%\% para
celdas "mágicas"

    \begin{Verbatim}[commandchars=\\\{\}]
{\color{incolor}In [{\color{incolor}4}]:} \PY{o}{\PYZpc{}}\PY{k}{magic}
\end{Verbatim}


    \begin{Verbatim}[commandchars=\\\{\}]
{\color{incolor}In [{\color{incolor}5}]:} \PY{o}{\PYZpc{}}\PY{k}{quickref}
\end{Verbatim}


    \subsection{1. Variables y tipos}\label{variables-y-tipos}

    En python se distinguen 3 tipo de variables: numéricas (int,float), tipo
texto(str) y lógicas (bool)

    \begin{verbatim}
1. Numéricas (enteros y reales)
\end{verbatim}

    \begin{Verbatim}[commandchars=\\\{\}]
{\color{incolor}In [{\color{incolor}21}]:} \PY{n}{a}\PY{o}{=}\PY{l+m+mi}{15}
         \PY{n+nb}{print}\PY{p}{(}\PY{n}{a}\PY{p}{)}
         \PY{n+nb}{print}\PY{p}{(}\PY{n}{b}\PY{p}{)}
\end{Verbatim}


    \begin{Verbatim}[commandchars=\\\{\}]
15
True

    \end{Verbatim}

    \begin{Verbatim}[commandchars=\\\{\}]
{\color{incolor}In [{\color{incolor}9}]:} \PY{n}{a}\PY{o}{=} \PY{l+m+mi}{5}
        \PY{n}{b}\PY{o}{=} \PY{l+m+mf}{6.5}
        
        \PY{c+c1}{\PYZsh{} Para imprimir una variable se usa la función print()}
        
        \PY{n+nb}{print}\PY{p}{(}\PY{n}{a}\PY{p}{)}
        \PY{n+nb}{print}\PY{p}{(}\PY{n}{b}\PY{p}{)}
\end{Verbatim}


    \begin{Verbatim}[commandchars=\\\{\}]
5
6.5

    \end{Verbatim}

    \begin{Verbatim}[commandchars=\\\{\}]
{\color{incolor}In [{\color{incolor}12}]:} \PY{n+nb}{print}\PY{p}{(}\PY{n+nb}{type}\PY{p}{(}\PY{n}{a}\PY{p}{)}\PY{p}{)}
\end{Verbatim}


    \begin{Verbatim}[commandchars=\\\{\}]
<class 'int'>

    \end{Verbatim}

    \begin{Verbatim}[commandchars=\\\{\}]
{\color{incolor}In [{\color{incolor}74}]:} \PY{c+c1}{\PYZsh{} Para conocer el tipo de la variable se usa la función type()}
         
         \PY{n+nb}{print}\PY{p}{(}\PY{n+nb}{type}\PY{p}{(}\PY{n}{a}\PY{p}{)}\PY{p}{,}\PY{n+nb}{type}\PY{p}{(}\PY{n}{b}\PY{p}{)}\PY{p}{)}
\end{Verbatim}


    \begin{Verbatim}[commandchars=\\\{\}]
<class 'int'> <class 'float'>

    \end{Verbatim}

    \begin{verbatim}
2. Texto
\end{verbatim}

    En la creación de las variables tipo texto se deben emplear las comillas
para indicar que es considerado como texto, de lo contratio se generará
un error

    \begin{Verbatim}[commandchars=\\\{\}]
{\color{incolor}In [{\color{incolor}13}]:} \PY{n}{Str} \PY{o}{=} \PY{n}{Hola}
\end{Verbatim}


    \begin{Verbatim}[commandchars=\\\{\}]

        ---------------------------------------------------------------------------

        NameError                                 Traceback (most recent call last)

        <ipython-input-13-d677e4aeba82> in <module>()
    ----> 1 Str = Hola
    

        NameError: name 'Hola' is not defined

    \end{Verbatim}

    \begin{Verbatim}[commandchars=\\\{\}]
{\color{incolor}In [{\color{incolor}77}]:} \PY{n}{Str} \PY{o}{=} \PY{l+s+s2}{\PYZdq{}}\PY{l+s+s2}{Hola mundo!}\PY{l+s+s2}{\PYZdq{}}
         \PY{n+nb}{print}\PY{p}{(}\PY{n}{Str}\PY{p}{)}
\end{Verbatim}


    \begin{Verbatim}[commandchars=\\\{\}]
Hola mundo!

    \end{Verbatim}

    \begin{Verbatim}[commandchars=\\\{\}]
{\color{incolor}In [{\color{incolor}14}]:} \PY{n}{Nombre}\PY{o}{=} \PY{n+nb}{input}\PY{p}{(}\PY{l+s+s2}{\PYZdq{}}\PY{l+s+s2}{Ingrese su nombre: }\PY{l+s+s2}{\PYZdq{}}\PY{p}{)}
         \PY{n+nb}{print}\PY{p}{(}\PY{l+s+s2}{\PYZdq{}}\PY{l+s+s2}{Bienvenido a OPALO }\PY{l+s+s2}{\PYZdq{}}\PY{o}{+} \PY{n}{Nombre}\PY{o}{+}\PY{l+s+s2}{\PYZdq{}}\PY{l+s+s2}{!}\PY{l+s+s2}{\PYZdq{}}\PY{p}{)}
\end{Verbatim}


    \begin{Verbatim}[commandchars=\\\{\}]
Ingrese su nombre: David
Bienvenido a OPALO David!

    \end{Verbatim}

    \begin{Verbatim}[commandchars=\\\{\}]
{\color{incolor}In [{\color{incolor}80}]:} \PY{n+nb}{type}\PY{p}{(}\PY{n}{Nombre}\PY{p}{)}
\end{Verbatim}


\begin{Verbatim}[commandchars=\\\{\}]
{\color{outcolor}Out[{\color{outcolor}80}]:} str
\end{Verbatim}
            
    \begin{verbatim}
3. Booleanas
\end{verbatim}

    \begin{Verbatim}[commandchars=\\\{\}]
{\color{incolor}In [{\color{incolor}15}]:} \PY{n}{a}\PY{o}{=} \PY{k+kc}{True}
         \PY{n}{b}\PY{o}{=} \PY{l+m+mi}{34}\PY{o}{\PYZgt{}}\PY{l+m+mi}{29}
         
         \PY{n+nb}{print}\PY{p}{(}\PY{n}{b}\PY{p}{)}
\end{Verbatim}


    \begin{Verbatim}[commandchars=\\\{\}]
True

    \end{Verbatim}

    \begin{Verbatim}[commandchars=\\\{\}]
{\color{incolor}In [{\color{incolor}16}]:} \PY{n+nb}{type}\PY{p}{(}\PY{n}{b}\PY{p}{)}
\end{Verbatim}


\begin{Verbatim}[commandchars=\\\{\}]
{\color{outcolor}Out[{\color{outcolor}16}]:} bool
\end{Verbatim}
            
    Para nombrar las variables se pueden asignar nombres libremente, sin
embargo, no pueden empezar con valores numéricos o utilizar palabras
reservadas (33 palabras), como:

\begin{array}{|c|c|} \hline
and &del &from &None &True \\ \hline
as &elif &global &nonlocal &try \\ \hline
assert &else &if &not &while \\ \hline
break &except &import &or &with \\ \hline
class &False &in &pass &yield \\ \hline
continue &finally &is &raise \\ \hline
def &for &lambda &return \\ \hline
\end{array}

    \subsection{2. Operadores}\label{operadores}

    Respecto a los operadores se distinguen tres tipos: * aritméticos: +, -,
/, \% * booleano: and, or, not * comparación: ==, \textless{}=,
\textgreater{}=, !=

Para utilizar los operadores aritméticos los valores deben ser del mismo
tipo, es decir, se suma texto con texto o número con número, en el caso
de los tipo bool el valor de True es igual a 1 y False a 0

    \begin{Verbatim}[commandchars=\\\{\}]
{\color{incolor}In [{\color{incolor}2}]:} \PY{n}{help}\PY{p}{(}\PY{l+s+s2}{\PYZdq{}}\PY{l+s+s2}{+}\PY{l+s+s2}{\PYZdq{}}\PY{p}{)} \PY{c+c1}{\PYZsh{} Obtener ayuda sobre los operadores}
\end{Verbatim}


    \begin{Verbatim}[commandchars=\\\{\}]
Operator precedence
*******************

The following table summarizes the operator precedence in Python, from
lowest precedence (least binding) to highest precedence (most
binding).  Operators in the same box have the same precedence.  Unless
the syntax is explicitly given, operators are binary.  Operators in
the same box group left to right (except for exponentiation, which
groups from right to left).

Note that comparisons, membership tests, and identity tests, all have
the same precedence and have a left-to-right chaining feature as
described in the Comparisons section.

+-------------------------------------------------+---------------------------------------+
| Operator                                        | Description                           |
+=================================================+=======================================+
| "lambda"                                        | Lambda expression                     |
+-------------------------------------------------+---------------------------------------+
| "if" -- "else"                                  | Conditional expression                |
+-------------------------------------------------+---------------------------------------+
| "or"                                            | Boolean OR                            |
+-------------------------------------------------+---------------------------------------+
| "and"                                           | Boolean AND                           |
+-------------------------------------------------+---------------------------------------+
| "not" "x"                                       | Boolean NOT                           |
+-------------------------------------------------+---------------------------------------+
| "in", "not in", "is", "is not", "<", "<=", ">", | Comparisons, including membership     |
| ">=", "!=", "=="                                | tests and identity tests              |
+-------------------------------------------------+---------------------------------------+
| "|"                                             | Bitwise OR                            |
+-------------------------------------------------+---------------------------------------+
| "\^{}"                                             | Bitwise XOR                           |
+-------------------------------------------------+---------------------------------------+
| "\&"                                             | Bitwise AND                           |
+-------------------------------------------------+---------------------------------------+
| "<<", ">>"                                      | Shifts                                |
+-------------------------------------------------+---------------------------------------+
| "+", "-"                                        | Addition and subtraction              |
+-------------------------------------------------+---------------------------------------+
| "*", "@", "/", "//", "\%"                        | Multiplication, matrix                |
|                                                 | multiplication, division, floor       |
|                                                 | division, remainder [5]               |
+-------------------------------------------------+---------------------------------------+
| "+x", "-x", "\textasciitilde{}x"                                | Positive, negative, bitwise NOT       |
+-------------------------------------------------+---------------------------------------+
| "**"                                            | Exponentiation [6]                    |
+-------------------------------------------------+---------------------------------------+
| "await" "x"                                     | Await expression                      |
+-------------------------------------------------+---------------------------------------+
| "x[index]", "x[index:index]",                   | Subscription, slicing, call,          |
| "x(arguments{\ldots})", "x.attribute"                | attribute reference                   |
+-------------------------------------------------+---------------------------------------+
| "(expressions{\ldots})", "[expressions{\ldots}]", "\{key:  | Binding or tuple display, list        |
| value{\ldots}\}", "\{expressions{\ldots}\}"                  | display, dictionary display, set      |
|                                                 | display                               |
+-------------------------------------------------+---------------------------------------+

-[ Footnotes ]-

[1] While "abs(x\%y) < abs(y)" is true mathematically, for floats
    it may not be true numerically due to roundoff.  For example, and
    assuming a platform on which a Python float is an IEEE 754 double-
    precision number, in order that "-1e-100 \% 1e100" have the same
    sign as "1e100", the computed result is "-1e-100 + 1e100", which
    is numerically exactly equal to "1e100".  The function
    "math.fmod()" returns a result whose sign matches the sign of the
    first argument instead, and so returns "-1e-100" in this case.
    Which approach is more appropriate depends on the application.

[2] If x is very close to an exact integer multiple of y, it's
    possible for "x//y" to be one larger than "(x-x\%y)//y" due to
    rounding.  In such cases, Python returns the latter result, in
    order to preserve that "divmod(x,y)[0] * y + x \% y" be very close
    to "x".

[3] The Unicode standard distinguishes between *code points* (e.g.
    U+0041) and *abstract characters* (e.g. "LATIN CAPITAL LETTER A").
    While most abstract characters in Unicode are only represented
    using one code point, there is a number of abstract characters
    that can in addition be represented using a sequence of more than
    one code point.  For example, the abstract character "LATIN
    CAPITAL LETTER C WITH CEDILLA" can be represented as a single
    *precomposed character* at code position U+00C7, or as a sequence
    of a *base character* at code position U+0043 (LATIN CAPITAL
    LETTER C), followed by a *combining character* at code position
    U+0327 (COMBINING CEDILLA).

    The comparison operators on strings compare at the level of
    Unicode code points. This may be counter-intuitive to humans.  For
    example, ""\textbackslash{}u00C7" == "\textbackslash{}u0043\textbackslash{}u0327"" is "False", even though both
    strings represent the same abstract character "LATIN CAPITAL
    LETTER C WITH CEDILLA".

    To compare strings at the level of abstract characters (that is,
    in a way intuitive to humans), use "unicodedata.normalize()".

[4] Due to automatic garbage-collection, free lists, and the
    dynamic nature of descriptors, you may notice seemingly unusual
    behaviour in certain uses of the "is" operator, like those
    involving comparisons between instance methods, or constants.
    Check their documentation for more info.

[5] The "\%" operator is also used for string formatting; the same
    precedence applies.

[6] The power operator "**" binds less tightly than an arithmetic
    or bitwise unary operator on its right, that is, "2**-1" is "0.5".

Related help topics: lambda, or, and, not, in, is, BOOLEAN, COMPARISON,BITWISE, SHIFTING, BINARY, FORMATTING, POWER, UNARY, ATTRIBUTES,SUBSCRIPTS, SLICINGS, CALLS, TUPLES, LISTS, DICTIONARIES


    \end{Verbatim}

    \begin{Verbatim}[commandchars=\\\{\}]
{\color{incolor}In [{\color{incolor}3}]:} \PY{c+c1}{\PYZsh{} Operadores artitméticos}
        
        \PY{n}{nombre}\PY{o}{=}\PY{l+s+s2}{\PYZdq{}}\PY{l+s+s2}{Grupo}\PY{l+s+s2}{\PYZdq{}}\PY{o}{+} \PY{l+s+s2}{\PYZdq{}}\PY{l+s+s2}{ }\PY{l+s+s2}{\PYZdq{}} \PY{o}{+}\PY{l+s+s2}{\PYZdq{}}\PY{l+s+s2}{OPALO}\PY{l+s+s2}{\PYZdq{}}
        \PY{n+nb}{print}\PY{p}{(}\PY{n}{nombre}\PY{p}{)}
        
        \PY{n+nb}{print}\PY{p}{(}\PY{l+m+mi}{5}\PY{o}{/}\PY{o}{/}\PY{l+m+mi}{3}\PY{p}{)} \PY{c+c1}{\PYZsh{} Realiza la aproximación al entero menor}
        \PY{n+nb}{print}\PY{p}{(}\PY{l+m+mi}{5}\PY{o}{\PYZpc{}}\PY{k}{3})  \PYZsh{} Modulo de la división
        \PY{n+nb}{print}\PY{p}{(}\PY{l+m+mi}{5}\PY{o}{*}\PY{o}{*}\PY{l+m+mi}{3}\PY{p}{)} \PY{c+c1}{\PYZsh{} Potenciación}
\end{Verbatim}


    \begin{Verbatim}[commandchars=\\\{\}]
Grupo OPALO
1
2
125

    \end{Verbatim}

    \begin{Verbatim}[commandchars=\\\{\}]
{\color{incolor}In [{\color{incolor}7}]:} \PY{c+c1}{\PYZsh{} Operadores lógicos y de comparación}
        
        \PY{n}{s}\PY{o}{=}\PY{l+m+mi}{15}\PY{o}{==}\PY{l+m+mi}{12}
        \PY{n}{a}\PY{o}{=} \PY{l+m+mi}{45}\PY{o}{==}\PY{l+m+mi}{45} \PY{o+ow}{and} \PY{l+m+mi}{13}\PY{o}{\PYZgt{}}\PY{l+m+mi}{9}
        \PY{n}{b}\PY{o}{=} \PY{l+m+mi}{15}\PY{o}{!=}\PY{l+m+mi}{12} \PY{o+ow}{or} \PY{l+m+mi}{12}\PY{o}{\PYZgt{}}\PY{l+m+mi}{9}
        \PY{n}{c}\PY{o}{=} \PY{l+m+mi}{15} \PY{o+ow}{not} \PY{o+ow}{in} \PY{p}{[}\PY{l+m+mi}{12}\PY{p}{,}\PY{l+m+mi}{14}\PY{p}{,}\PY{l+m+mi}{16}\PY{p}{]}
        
        \PY{n+nb}{print}\PY{p}{(}\PY{n}{a}\PY{p}{)}
        \PY{n+nb}{print}\PY{p}{(}\PY{n}{b}\PY{p}{)}
        \PY{n+nb}{print}\PY{p}{(}\PY{n}{c}\PY{p}{)}
\end{Verbatim}


    \begin{Verbatim}[commandchars=\\\{\}]
True
True
True

    \end{Verbatim}

    Para python el orden de las operaciones sigue la convención matemática
(PEMDAS)

\begin{enumerate}
\def\labelenumi{\arabic{enumi}.}
\tightlist
\item
  Paréntesis
\item
  Potenciación (Exponentiation)
\item
  Multiplicación y división
\item
  Suma y resta
\end{enumerate}

Operadores con la misma presedencia son evaluados de izquierda a derecha

    \begin{Verbatim}[commandchars=\\\{\}]
{\color{incolor}In [{\color{incolor}1}]:} \PY{l+m+mi}{5}\PY{o}{\PYZhy{}}\PY{l+m+mi}{2}\PY{o}{*}\PY{l+m+mi}{5}\PY{o}{\PYZhy{}}\PY{l+m+mi}{1}
\end{Verbatim}


\begin{Verbatim}[commandchars=\\\{\}]
{\color{outcolor}Out[{\color{outcolor}1}]:} -6
\end{Verbatim}
            
    \begin{Verbatim}[commandchars=\\\{\}]
{\color{incolor}In [{\color{incolor}4}]:} \PY{p}{(}\PY{l+m+mi}{5}\PY{o}{\PYZhy{}}\PY{l+m+mi}{2}\PY{p}{)}\PY{o}{*}\PY{l+m+mi}{5}\PY{o}{\PYZhy{}}\PY{l+m+mi}{1}
\end{Verbatim}


\begin{Verbatim}[commandchars=\\\{\}]
{\color{outcolor}Out[{\color{outcolor}4}]:} 14
\end{Verbatim}
            
    \begin{Verbatim}[commandchars=\\\{\}]
{\color{incolor}In [{\color{incolor}5}]:} \PY{l+m+mi}{5}\PY{o}{\PYZhy{}}\PY{l+m+mi}{2}\PY{o}{*}\PY{p}{(}\PY{l+m+mi}{5}\PY{o}{\PYZhy{}}\PY{l+m+mi}{1}\PY{p}{)}
\end{Verbatim}


\begin{Verbatim}[commandchars=\\\{\}]
{\color{outcolor}Out[{\color{outcolor}5}]:} -3
\end{Verbatim}
            
    \subsubsection{Ejercicios}\label{ejercicios}

\begin{enumerate}
\def\labelenumi{\alph{enumi}.}
\item
  Cree un notebook y responda a cada uno de los siguientes items en
  celdas independientes.
\item
  Cree una variable de cada tipo e imprima su tipo
\item
  A que se debe el error cuando se generea la siguiente variable

  var = "OPALO" + 205
\item
  Ubique los paréntesis de tal manera que la operación sea igual a "0"

\begin{verbatim}
8 - 3* 2 - 1 + 1
\end{verbatim}
\item
  ¿Cuál es el valor de la suiguiente expresión? \# 7-\/-\/-\/-\/-\/-3
\end{enumerate}

    \begin{Verbatim}[commandchars=\\\{\}]
{\color{incolor}In [{\color{incolor}8}]:}  \PY{l+m+mi}{8} \PY{o}{\PYZhy{}} \PY{p}{(}\PY{l+m+mi}{3}\PY{o}{*} \PY{l+m+mi}{2}\PY{p}{)} \PY{o}{\PYZhy{}} \PY{p}{(}\PY{l+m+mi}{1} \PY{o}{+} \PY{l+m+mi}{1}\PY{p}{)}
\end{Verbatim}


\begin{Verbatim}[commandchars=\\\{\}]
{\color{outcolor}Out[{\color{outcolor}8}]:} 0
\end{Verbatim}
            

    % Add a bibliography block to the postdoc
    
    
    
    \end{document}
